\documentclass{article}
\usepackage{amsmath,amssymb}
\usepackage{color}
\usepackage{bm}
\usepackage[sort&compress]{natbib}
\usepackage[normalem]{ulem}	% Part of the standard distribution


\begin{document}
\textbf{Question:}
\begin{quotation}
The main quantity that the paper focuses on is the thickness of the film between the bubbles and the
wall. As shown by Figure 8 and as mentioned by the Authors in Section 4, the film thickness varies
along the bubble. However, the Authors do not explain the choice of the location of the measurement
of the film thickness. This choice is of primary importance when comparing results with those of
the
literature and in particular with works on finger propagation in which the film thickness is
measured
”at infinity”.
\end{quotation}

\textbf{Answer:} The following section is added to the text 
\begin{quotation}
We chose to measure the film thickness in the middle of the bubble. The
classical Bretherton formulation is for the measurement of the film thickness at ``infinity''.
However, \citet{cerro-bubble-train} indicate that for the gravity-driven bubble train flow in tubes
it is enough for bubble to have a length of two-three diameters to apply analysis for very long
bubbles. In comparison with the axisymmetric circular geometry, for three-dimensional square shaped
microchannels \citet{heil-threedim} indicate the measurement of the film thickness at $5.5$
channel heights from the front bubble tip. This distance was shown sufficient for the
film thickness to establish itself for the capillary number range $Ca<4$. We chose to measure
the film thickness in the middle of the bubble, which is located at least at the distance of $2.5-3$
channel heights from the front bubble tip.  Moreover, the film thickness examination, see Section
\ref{section:film:thickness:variation}, shows that for the capillary range of interest, i.e.
$0.08\leq Ca \leq 1$, the standard deviation from the bubble middle film thickness is around $7$
 percent for small capillary numbers ($Ca=0.05$) and less than $1$ percent for larger ones
($Ca \geq 0.1$) at the distances ranging from one channel height from the rear tip upto one channel
height to the front tip. This certainly validates the point of choice to measure bubble thickness.
\end{quotation}
 
\textbf{Question:}
\begin{quotation}
It may be appropriate to introduce Figure 8 on thickness variation before the main results on film
thickness.
\end{quotation}

\textbf{Answer:} We agree and moved the section before the results on the capillary number
dependency.


\textbf{Question:}
\begin{quotation}
The lines 54-55 indicating that ”numerical simulations and experimental studies showed [...]
Reynolds
number effects on the film thickness for capillary numbers larger than 0.003” is in contradiction
with
the lines 32-33 which state ”negligible Reynolds number effects on the film thickness for a
relatively
wide range of Reynolds numbers.”
\end{quotation}

\textbf{Answer:} The following paragraph is added:
\begin{quotation}
Moreover, the results of \citet{giavedoni-numerical} and \citet{heil-bretherton} show
{\color{red} insignificant} Reynolds number effects on the film thickness for a relatively wide
range of Reynolds
numbers. 
{\color{red} For example, \citet{giavedoni-numerical} suggested that the Reynolds number effects
are negligible for $Ca\leq0.05$ and have moderate changes for $Ca>0.05$ in the range of Reynolds
number from $0$ to $70$. Later on, citet{heil-bretherton} extended results upto Reynolds number
$300$. The indicate that while the influence of the established film thickness is insignificant
($7$ percent from the film thickness measured at $Re=0$). However, the Reynolds number influence
significantly the pressure distrubution and the flow field near the front bubble tip. In
the present simulations the largest Reynolds number we achieve is less than $20$ and it allows to
neglect the
inertia effects.} 

\citet{kreutzer-pressure-drop} showed a deviation from the $Ca^{2/3}$ rule
{\color{red}\sout{and Reynolds number 
effects on the film thickness}} for capillary numbers larger than $0.003$.
\end{quotation}

\textbf{Question:}
\begin{quotation}
Although inertia has been shown to have a minor effect on film thickness, inertia does influence
some flow variables in the Bretherton’s problem (see Ref. 3 for instance). To check whether inertia
is important or not, and to discuss the assumption of uniform density, the Authors should provide
the reader with the Reynolds numbers associated with each simulation. This is also required for the
consistency of Figure 7 in which results are compared with other works of the literature.
\end{quotation}

\textbf{Answer:} Thank you for pointing it out. We included the discussions of the Reynolds
effects in the answer to the question above. As well, we added Reynolds numbers to the Table for
the capillary range regime and the grid depedency study. 

\textbf{Question:}
\begin{quotation}
In this work, the diphasic flow is driven by a body force. The body force is therefore the control
parameter in the simulations. The Authors should give more detail about this body force. Indeed,
although the estimation of the body force from the pressure gradient is well described, it is not
clear
where this body force appears in the governing equations (Equations 2, 3 and 4). Does this body
force correspond to the variable $F_i$ in Equation 2? Should not this force or its equivalent appear
in
the Navier-Stokes equations (Equation 4)?
\end{quotation}

\textbf{Answer:} We updated equation (4) to include the force contribution. As well, the definition
of $P_{\alpha\beta}$ is added. $F_i$ is the force population which corresponds to the force
inclusion $\bm{F}$. We added the citation which describes how the force population is formed and
mimics the force behavior.

\textbf{Question:}
\begin{quotation}
I would suggest to calculate the Bond number to check if the shapes of the bubbles and the main
results of the paper are not influenced.
\end{quotation}

\textbf{Answer:} 
The maximum Bond number $Bo=\frac{\rho g N_y^2}{\gamma}$ in the
current simulation equals to $7$, which is less than the number when the shape of the bubble is
affected \cite{zheng-large-ratio}. However, the point is that the equation of state in the lattice
Boltzmann framework is connected with density. Thus usually either pressure (i.e. density) boundary
conditions are prescribed or the body force. If they are prescribed simultaneously then
we can consider the influence of the Bond number on the simulations. Otherwise prescribing just the
body force is analogous to prescribing the pressure gradient $\frac{\mathrm{d}P}{\mathrm{d}x}$ in
the Navier-Stokes equation. Thus, it's not possbile the Bond number effect. It is because of the
coupled nature of the pressure and density in the LBM framework. 

\textbf{Question:}
The use of the variable $Ca_{lit}$ and the calculation of the corresponding pressure gradient yield
estima
tions of the body force to apply and of the film thickness to choose grid resolution. The
presentation
of the corresponding procedure is repeated twice in Section 5-1 for $Ca=0.005$ and $Ca=0.05$. It is
again described in a different way in Section 5-4. Similar equations for the calculation of the
pressure
gradient are thus given in Equations 6, 14 and 16. The Authors should summarize these equations
into one unless they explain their differences. In fact, it would probably be straightforward not to
detail the calculation of estimated variables since they are intermediary variables which allow the
design of the benchmark but do not influence directly the results. Also, the Authors could possibly
say they control and impose directly the body force without describing all the estimation procedure.

\textbf{Answer:} Thank you for the suggestion. While we agree that there is redundant information
about initialization and this was taken into the account by reducing repetitive text:
\begin{quotation}
\end{quotation}

 However, the
dimensionalization of the problem and proper benchmarking is quite difficult in the lattice
Boltzmann framework. You can refer to the lbmethod.org/forum where half of all questions related to
the lattice Boltzmann are refered to the non-dimensionalization procedure. Therefore, we decided to
keep equation 16 as the most informative for the LBM community and to reduce {\color{red} equation
14 and 6}.
\bibliographystyle{unsrtnat}
\bibliography{answers}

\end{document}
